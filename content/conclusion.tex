\section{Conclusions and Future Works}
Usage management is a common problem set with features embodied in domains ranging from security application to video games to music production and retail.  The ability to provide management of resources with regard to authorized subjects is being addressed in multiple different forums, many of which are taking remarkably different approaches.  Common features however generally include the need for either ubiquitous rights expression language acceptance or for extensive translation from between all supported rights languages.

In this paper, we first demonstrated the development of the initial model we used to define our problem space.  Here, we described the general use of the system, who the primary users were, what the expected lifecyle of policies was, and what the domain model looked like.  We then implemented the syntax of the DSL, in Ruby, as an internal DSL with specific examples.  We wrapped up the paper with demonstrations of equivalence to common rights management frameworks like the creative commons, ODRL, and XrML.

We have only begun to specify and use this particular DSL.  Future focus four our group on this effort will include additional language elaboration, exploration, and use in specific scenarios.  We need to spend additional time engineering the underlying software as well, so we can ensure that policies are in fact platform and environment agnostic, portable, and executable.  Finally, this implementation is an internal DSL within the Ruby language; we need to explore the application of external DSL techniques to this domain to better understand the required compromises between expressiveness and development difficulty.