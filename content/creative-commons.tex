\subsection{Creative Commons}\label{sec:model-cc}
In this section, we use our DSL to implement a policy that can be used with
Creative Commons licensed content.  The Creative Commons is a nonprofit
organization that provides free, easy-to-use legal tools that provide a simple,
standardized way to pre-clear usage rights to creative works for copyright
owners.  It gives copyright holders an easy way to license their works in a way
that leaves ``some rights reserved'' as compared to the ``all rights reserved''
default.  By providing these legal tools, the Creative Commons hopes to
increase the amount of creativity available ``in the commons'' \cite{creative-commons}.

Seven main license types are defined by Creative Commons:

\begin{itemize}
\item \textbf{Public Domain}. This license does not put any restrictions on how others may remix, tweak, or
build upon your work.
\item \textbf{Attribution}.  This license lets others distribute, remix, tweak, and build upon your work,
even commercially, as long as they credit you for the original creation. This
is the most accommodating of the licenses offered.  It is recommended for
maximum dissemination and use of licensed materials.
\item \textbf{Attribution, Share-Alike}.  This license lets others remix, tweak, and build upon your work even for
commercial purposes, as long as they credit you and license their new creations
under the identical terms.  This license is often compared to free and open
source software licenses.  All new works based on yours will carry the same
license, so any derivatives will also allow commercial use.
\item \textbf{Attribution, No-Derivatives}.  This license allows for redistribution, commercial and non-commercial, as long
as it is passed along unchanged and in whole, with credit to you.
\item \textbf{Attribution, Non-Commercial}.  This license lets others remix, tweak, and build upon your work
non-commercially, and although their new works must also acknowledge you and be
non-commercial, they do not have to license their derivative works on the same
terms.
\item \textbf{Attribution, Non-Commercial, Share-Alike}.  This license lets others remix, tweak, and build upon your work
non-commercially, as long as they credit you and license their new creations
under the identical terms.
\item \textbf{Attribution, Non-Commercial, No-Derivatives}.  This license is the most restrictive of the licenses, only allowing others to
download your works and share them with others as long as they credit you, but
they cannot change them in any way or use them commercially.
\end{itemize}

Now that we have explained the different Creative Commons license types, what
they mean, and the restrictions they have, we will create a policy with our DSL
that represents compliant usage of Creative Commons licensed works.

First, we use our DSL to define an activity.  In Creative Commons there is
really only one activity --  sharing a Creative Commons licensed work -- so
this code is fairly simple, defining the \texttt{share} activity.

\lstinputlisting[]{content/code/cc/activity.rb}

Next, we define constraints on the \texttt{share} activity.  There are four
such constraints in Creative Commons, as discussed above: attribution,
commercial use, derivative work, and share-alike.  This results in four
constraints defined in our DSL as follows:

\lstinputlisting[]{content/code/cc/constraints.rb}

Finally, we define the restricted activities, covering all seven basic license
types.  For example, in the following code listing, the
\texttt{share\_by\_work} restricted activity is defined by associating the
\texttt{attribution} constraint with the \texttt{share} activity. Likewise, the
\texttt{share\_by\_sa\_work} restricted activity associates the
\texttt{attribution} and \texttt{share\_alike} constraints with the
\texttt{share} activity.  Sharing activites for all seven license types are defined
similarly, resulting in a policy that allows for sharing all types of Creative
Commons licensed works, with proper constraints.

\lstinputlisting[]{content/code/cc/restrictedactivities.rb}

Now that we have implemented basic policy protections as defined in the creative commons environment, we will move on to ODRL.