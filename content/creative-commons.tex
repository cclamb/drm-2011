\subsection{Creative Commons}\label{sec:model-cc}
Creative commons applied\subsection{Creative Commons}\label{sec:model-cc}

The Creative Commons is a nonprofit organization that works to
increase the amount of creativity available in ``the commons''.  It provides
free, easy-to-use legal tools that provide a simple, standardized way to
pre-clear usage rights to creative work for copyright owners.  It lets
copyright holders easily change from default of ``all rights reserved'' to
``some rights reserved''.

In this section, we use our DSL to implement a policy that can be used with
Creative Commons licensed content, covering all seven of the main Creative
Commons license types: 

\begin{itemize}
\item \textit{Public Domain}.
\item \textit{Attribution}.
\item \textit{Attribution, Share-Alike}.
\item \textit{Attribution, No-Derivatives}.
\item \textit{Attribution, Non-Commercial}.
\item \textit{Attribution, Non-Commercial, Share-Alike}.
\item \textit{Attribution, Non-Commercial, No-Derivatives}.
\end{itemize}

%\begin{enumerate}
  %\item Public Domain
  %\item Attribution
  %\item Attribution, Share-Alike
  %\item Attribution, No-Derivatives
  %\item Attribution, Non-Commercial
  %\item Attribution, Non-Commercial, Share-Alike
  %\item Attribution, Non-Commercial, No-Derivatives
%\end{enumerate}

First, we create an activity.  This activity represents sharing a Creative
Commons licensed asset.

\lstinputlisting[]{content/code/cc/activity.rb}

Next, we define constraints on this share activity.  There are four such
constraints in Creative Commons: attribution, commercial use, derivative work,
and share-alike.

\lstinputlisting[]{content/code/cc/constraints.rb}

Finally, we define the restricted activities, covering all 7 basic license
types.

\lstinputlisting[]{content/code/cc/restrictedactivities.rb}