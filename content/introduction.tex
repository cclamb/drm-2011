\section{Introduction}
For the purposes of this paper, we current define usage management as the management of the usage of resources (and data) across and within computing environments.  More than access control or digital rights management, usage management concerns itself with fine-grained control of all aspects of how a given digital resource is used.  As digital environments become more open over time, the need for usage management for resources that span utility computational environments (e.g. cloud provider systems) will become increasingly important.

With the advent and widespread use of cloud computing, those responsible for a given usage managed resource are almost never those responsible for the computing systems, except at edge devices like mobile phones or other small profile computing devices.  Resources are regularly moved across national boundaries and regional areas without either the content owner's or creator's knowledge.  Furthermore, this kind of transfer is generally according to pre-established algorithms or data routing protocols over which users of all stripes have no control.  Managing these kinds of issues requires new usage management capabilities that can run on platforms ranging from small, hand-held devices to nodes in large data centers.

Historically research in this area has been focused on developing more expressive policy languages via either different type of mathematical logics or formalisms with greater reasoning capabilities~\cite{ArHu:07,BaMi:06,ChCoEtHaJoLa:03,HaWe:04,HaWe:08,PuWe:02,XiBjFu:08}.  These approaches however fail to address interoperability challenges posed by new commercially available distributed computing environments.Interoperability efforts have resorted to translation mechanisms, where the policy is translated in its entirety to a different language~\cite{HeJa:05,PoPrDe:04,ScTaWo:04}; it has been shown recently however that such techniques are infeasible and hard to perform for most policy languages~\cite{KoLaMaMi:04, SaShUe:04}. Other approaches have led to complex policy specification languages that have tried to establish themselves as the universal standard~\cite{OMADRM,ODRL-req,Wa:04,XrML-spec}.  This unfortunately tends to stifle both innovation and flexibility~\cite{HeJa:05,JaHe:04,JaHe:08,JaHeMa:06}.

To address these issues, we first applied the principles of system design to develop a framework for usage management in open, distributed environments that supports interoperability. These principles have been used by researchers in large network design create a balance between interoperability and open, flexible architectures~\cite{Al:04,BlCl:01,ClWrSoBr:02}, allowing for computing scale and power without sacrificing innovation. Initially we standardized certain features of the framework operational semantics, and left free of standards features that necessitate choice and innovation.

We have implemented this framework, including a usage management calculus providing a platform usage management, within a Domain Specific Language (DSL) and associated evaluation environment. The DSL and its environment implements our previously defined framework, separating various roles needed for distributed policy creation and management, provides the capability to develop executable licenses, and is extensible from both a policy and constraint definition perspective.

In this paper, in Section \ref{sec:model} we will first review the model we developed to guide the DSL's syntactic and semantic development.  Then, in the next section, we will cover the language itself, how it was developed, and its supporting evaluation environment.  We will then close the paper with three specific implemenation examples showing how the language and its runtime support usage management scenarios from three different environments --- creative commons (CC), the extensible rights markup language (XRML), and the open digital rights language(ODRL).

\subsection{Previous Work}
Lorem ipsum dolor sit amet, consectetur adipiscing elit. Curabitur vitae dolor elit, vel sagittis justo. Nullam vehicula scelerisque fermentum. Quisque pulvinar, neque sit amet tempor sagittis, risus felis consequat massa, at euismod quam lacus sed sapien. Integer nec viverra mi. Mauris ultricies tellus non nisl tincidunt dictum. Pellentesque pretium lectus consectetur mauris tempor rutrum. Pellentesque habitant morbi tristique senectus et netus et malesuada fames ac turpis egestas. Duis dictum quam tellus, eu scelerisque elit. Class aptent taciti sociosqu ad litora torquent per conubia nostra, per inceptos himenaeos. Nullam massa leo, commodo id suscipit eu, consectetur at quam. Cum sociis natoque penatibus et magnis dis parturient montes, nascetur ridiculus mus. Etiam eu diam leo, in rutrum justo. Phasellus ut turpis at orci tempor varius. Suspendisse iaculis faucibus bibendum. Etiam eget mollis nunc. Suspendisse potenti. Pellentesque lectus leo, ornare a ultrices a, lacinia in diam. Vivamus eu justo at lacus sodales eleifend consequat id dui. Nullam laoreet rhoncus malesuada. Curabitur sit amet cursus diam.

In hac habitasse platea dictumst. Suspendisse potenti. Nunc eros libero, eleifend eu scelerisque sit amet, varius rutrum felis. Nulla facilisi. Phasellus tincidunt sapien a libero blandit blandit. Fusce rutrum aliquet lacus, non rhoncus risus facilisis at. Mauris semper varius quam et placerat. Fusce sed suscipit mi. Vivamus a ligula ante, in adipiscing velit. Nullam auctor molestie tincidunt. Duis blandit diam et ante congue vitae laoreet elit sodales. Proin eu nulla vitae est tincidunt rhoncus. Duis at ultrices odio.